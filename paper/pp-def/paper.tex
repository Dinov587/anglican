\documentclass[a4paper,oneside]{article}
\usepackage[]{geometry}
% Other packages
\usepackage{color}
\usepackage{amsmath}
\usepackage{amssymb}
\usepackage{amsthm}
\usepackage{hyperref}
\usepackage[ruled]{algorithm}
\usepackage{algpseudocode}
\usepackage{graphicx}
\usepackage{colortbl}
\usepackage{arydshln}
\usepackage{minted}

% illustrations
\usepackage{tikz}
\usetikzlibrary{positioning}
\usetikzlibrary{shapes.geometric}
\usetikzlibrary{shapes.symbols}
\usetikzlibrary{shadows}
\usetikzlibrary{arrows}

\newcommand{\diff}[1] {{\color{blue} #1}}

\renewcommand{\topfraction}{0.9}

\newtheorem{dfn}{Definition}
\newtheorem{thm}{Theorem}
\newtheorem{lmm}{Lemma}
\newtheorem{crl}{Corollary}

\title{Probabilistic Program}
\author {}

\begin{document}

\maketitle

\section*{Intuition}

Probabilistic programs are regular programs extended by two
constructs~\cite{GHNR14}:
\begin{itemize}
	\item The ability to draw random values from probability
		distributions.
	\item The ability to condition values computed in the
		programs on probability distributions.
\end{itemize}
A probabilistic program implicitly defines a probability
distribution over program output.

\section*{Definition}

A probabilistic program is a stateful deterministic
computation $\mathcal{P}(\theta)$ with the following properties:

\begin{itemize}
	\item Initially, $\mathcal{P}$ expects a value of $\theta$
		as the argument.
	\item On every invocation, $\mathcal{P}$ returns either a
		distribution $F$, a distribution and a value $(G, y)$, a
		value $z$, or $\bot$.
	\item Upon returning $F$, $\mathcal{P}$ expects a value $x$
		drawn from $F$ as the argument to continue.
	\item Upon returning $(G, y)$ or $z$, $\mathcal{P}$ is
		invoked again without arguments.
	\item Upon returning $\bot$, $\mathcal{P}$ terminates.
\end{itemize}

A program is run by calling $\mathcal{P}$ repeatedly until
termination. Every run of the program implicitly produces a
sequence of pairs $(F_i, x_i)$ of distributions and values drawn
from them. We call this sequence a \textit{trace} and denote it
by $\pmb{x}$.  A trace induces a sequence of pairs $(G_j, y_j)$
of observed random variables and their values.  We call this
sequence an \textit{image} and denote it by $\pmb{y}$. We call a
sequence of values $z_k$ an \textit{output} of the program and
denote it by $\pmb{z}$.  Program output is deterministic given
the trace.

\emph{By definition,} the probability of a trace is proportional
to the product of the probability of all random choices
$\pmb{x}$ and the likelihood of all observations $\pmb{y}$:
\begin{equation}
	p_{\mathcal{P}}(\pmb{x}|\theta,\pmb{y}) \propto \prod_{i=1}^{\left|\pmb{x}\right|} p_{F_i}(x_i) \prod_{j=1}^{\left|\pmb{y}\right|}p_{G_j}(y_{j})
  \label{eqn:p-trace}
\end{equation}
The objective of inference in probabilistic program
$\mathcal{P}$ is to discover the distribution
$p_{\mathcal{P}}(\pmb{z}|\theta)$ of program output $\pmb{z}$.

\section*{Implementation}

Several implementations of general probabilistic programming
languages are available~\cite{GMR+08,MSP14,WVM14} and the list
is growing: \url{http://probabilistic-programming.org/wiki/Home}.
Inference is usually performed using Monte Carlo sampling
algorithms for probabilistic programs~\cite{WSG11,WVM14,PWD+14}.
While some algorithms are better suited for certain problem
types, most can be used with any valid probabilistic program.

\bibliographystyle{alpha}
\bibliography{refs}

\end{document}
